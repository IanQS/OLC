%%%%%%%%%%%%%%%%%%%%%%%%%%%%%%%%%%%%%%%%%
% Structured General Purpose Assignment
% LaTeX Template
%
% This template has been downloaded from:
% http://www.latextemplates.com
%
% Original author:
% Ted Pavlic (http://www.tedpavlic.com)
%
% Note:
% The \lipsum[#] commands throughout this template generate dummy text
% to fill the template out. These commands should all be removed when
% writing assignment content.
%
%%%%%%%%%%%%%%%%%%%%%%%%%%%%%%%%%%%%%%%%%

%----------------------------------------------------------------------------------------
%	PACKAGES AND OTHER DOCUMENT CONFIGURATIONS
%----------------------------------------------------------------------------------------

\documentclass{article}

\usepackage{fancyhdr} % Required for custom headers
\usepackage{lastpage} % Required to determine the last page for the footer
\usepackage{extramarks} % Required for headers and footers
\usepackage{graphicx} % Required to insert images
\usepackage{lipsum} % Used for inserting dummy 'Lorem ipsum' text into the
\usepackage{ textcomp }

\usepackage{hyperref}
\hypersetup{
    colorlinks=true,
    linkcolor=blue,
    filecolor=magenta,
    urlcolor=cyan,
}
% template

\usepackage{amsmath}

% Margins
\topmargin=-0.45in
\evensidemargin=0in
\oddsidemargin=0in
\textwidth=6.5in
\textheight=9.0in
\headsep=0.25in

\linespread{1.1} % Line spacing

% Set up the header and footer
\pagestyle{fancy}
\lhead{\hmwkAuthorName} % Top left header
\chead{\hmwkClass\ (\hmwkTitle)} % Top center header
\rhead{\firstxmark} % Top right header
\lfoot{\lastxmark} % Bottom left footer
\cfoot{} % Bottom center footer
\rfoot{Page\ \thepage\ of\ \pageref{LastPage}} % Bottom right footer
\renewcommand\headrulewidth{0.4pt} % Size of the header rule
\renewcommand\footrulewidth{0.4pt} % Size of the footer rule

\setlength\parindent{0pt} % Removes all indentation from paragraphs

%----------------------------------------------------------------------------------------
%	DOCUMENT STRUCTURE COMMANDS
%	Skip this unless you know what you're doing
%----------------------------------------------------------------------------------------

% Header and footer for when a page split occurs within a problem environment
\newcommand{\enterProblemHeader}[1]{
\nobreak\extramarks{#1}{#1 continued on next page\ldots}\nobreak
\nobreak\extramarks{#1 (continued)}{#1 continued on next page\ldots}\nobreak
}

% Header and footer for when a page split occurs between problem environments
\newcommand{\exitProblemHeader}[1]{
\nobreak\extramarks{#1 (continued)}{#1 continued on next page\ldots}\nobreak
\nobreak\extramarks{#1}{}\nobreak
}

\setcounter{secnumdepth}{0} % Removes default section numbers
\newcounter{homeworkProblemCounter} % Creates a counter to keep track of the number of problems

\newcommand{\homeworkProblemName}{}
\newenvironment{homeworkProblem}[1][Topic \arabic{homeworkProblemCounter}]{ % Makes a new environment called homeworkProblem which takes 1 argument (custom name) but the default is "Problem #"
\stepcounter{homeworkProblemCounter} % Increase counter for number of problems
\renewcommand{\homeworkProblemName}{#1} % Assign \homeworkProblemName the name of the problem
\section{\homeworkProblemName} % Make a section in the document with the custom problem count
\enterProblemHeader{\homeworkProblemName} % Header and footer within the environment
}{
\exitProblemHeader{\homeworkProblemName} % Header and footer after the environment
}

\newcommand{\problemAnswer}[1]{ % Defines the problem answer command with the content as the only argument
\noindent\framebox[\columnwidth][c]{\begin{minipage}{0.98\columnwidth}#1\end{minipage}} % Makes the box around the problem answer and puts the content inside
}

\newcommand{\homeworkSectionName}{}
\newenvironment{homeworkSection}[1]{ % New environment for sections within homework problems, takes 1 argument - the name of the section
\renewcommand{\homeworkSectionName}{#1} % Assign \homeworkSectionName to the name of the section from the environment argument
\subsection{\homeworkSectionName} % Make a subsection with the custom name of the subsection
\enterProblemHeader{\homeworkProblemName\ [\homeworkSectionName]} % Header and footer within the environment
}{
\enterProblemHeader{\homeworkProblemName} % Header and footer after the environment
}

%----------------------------------------------------------------------------------------
%	NAME AND CLASS SECTION
%----------------------------------------------------------------------------------------

\newcommand{\hmwkClass}{Compilers} % Course/class
\newcommand{\hmwkTitle}{Week 2} % Assignment title
\newcommand{\hmwkClassTime}{-} % Class/lecture time
\newcommand{\hmwkAuthorName}{Ian Quah (itq)} % Your name

%----------------------------------------------------------------------------------------
%	TITLE PAGE
%----------------------------------------------------------------------------------------

\title{
\vspace{2in}
\textmd{\textbf{\hmwkClass:\ \hmwkTitle}}\\
\vspace{3in}
}

\author{\textbf{\hmwkAuthorName}}

%----------------------------------------------------------------------------------------

\begin{document}

\maketitle
\newpage
%----------------------------------------------------------------------------------------
%	PROBLEM 1
%----------------------------------------------------------------------------------------

% To have just one problem per page, simply put a \clearpage after each problem

\begin{homeworkProblem}{Lexical Analysis - Introduction}

\problemAnswer{ % Answer
\begin{enumerate}
\item String $\rightarrow$ Lexical analyzer $\rightarrow$  (\textlangle class, string\textrangle
  \textbf{a.k.a} Token) $\rightarrow$ parser
\item \textbf{E.g}: foo = 42 $\rightarrow$ $<$ID, ``FOO'' $>$ $<$ OP, ``=''$>$ $<$
  Int,``42''$>$
  \item \begin{enumerate}
    \item \textbf{Operators}
    \item Whitespace
    \item Keywords
    \item Identifiers
    \item Numbers
    \item ( - open parens
    \item ) - closed parens
    \item ; - semi-colon
    \end{enumerate}
    The last 3 are often token classes of their own
\end{enumerate}
}
\end{homeworkProblem}

\begin{homeworkProblem}{Lexical Analysis - Examples}

\problemAnswer{ % Answer
\begin{enumerate}
\item \textbf{Lookahead}

  Necessary to disambiguate certain cases: might be the case that only much
  later in the program two cases are different.

  Necessary to decide where one token ends and next begins.

  Always necessary but we want to bound it as it increases complexity like crazy

\item \textbf{Goal}

  Partition input string into lexemes

  identify the token of each lexeme

\item Left-to-right scan makes lookahead necessary
\end{enumerate}
}
\end{homeworkProblem}

\newpage
\begin{homeworkProblem}{Lexical Analysis - Regular Languages}

\problemAnswer{ % Answer
\begin{enumerate}
\item Regular Languages:

  \begin{enumerate}
  \item Generally use Regexp to define them
  \item Specifies what set of languages are in a token class
  \item Single char:

    'c' = \{ ``c''\}: a language containing just the single character ``c''
  \item
  \item $\epsilon$

    $\epsilon$ = \{``''\}: the language containing the empty string. Is \textbf{NOT}
    null set

  \item union

    A + B = \{a $|$ a $\in$ A\} $\cup$ \{b $|$ b $\in$ B\}

  \item Concat

    AB = \{ab $|$ a $\in$ A $\wedge$  b $\in$ B \}

  \item Iteration a.k.a Kleene closure

    A* = $\cup_{i \geq 0} A^i$ \hspace{1 cm} A with itself i times.

    A$^0$ == $\epsilon$
  \end{enumerate}
\item Regular expressions over $\Sigma$ are the smallest set of expressions including:

  R = $\epsilon$

  $|$ 'c' s.t c $\in$ $\Sigma$

  $|$ R + R

  $|$ RR

  $|$ R$^*$

  All this is known as a grammar

\item Consider a language with the alphabet 0, 1 s.t: $\Sigma$ = \{0, 1\}
  \begin{enumerate}
  \item   1$^*$ = \hfill ``'', 1, 11, 111,....
  \item  (1 + 0) 1 = \{ab | a $\in$ $\wedge$ b $\in$ B\} = \hfill \{11, 10\}
  \item   (1 + 0)$^*$ 1 = ``'', 0 + 1, (1 + 0)(1 + 0), ... \hfill $\Sigma^*$

    $\Sigma^*$ is the set of all strings you can form out of the alphabet
  \end{enumerate}

\end{enumerate}
}
\end{homeworkProblem}

\newpage
\begin{homeworkProblem}{Lexical Analysis - Formal Languages}

\problemAnswer{ % Answer
\begin{enumerate}
\item \textbf{Definition: Formal Language}

  Let $\Sigma$ be a set of characters(an alphabet)

  A language over $\Sigma$ is a set of strings drawn from that alphabet

  difference between
  \href{https://cs.stackexchange.com/questions/55013/what-is-the-difference-between-formal-language-regular-language-and-regular-exp}{formal,
    regular, language, expression}

\item \textbf{Definition: Meaning Function, L}
  Using the regexp notation from earlier:

  \begin{enumerate}
  \item L: expr $\rightarrow$ Sets of Strings
  \item L('c') = \{ ``c''\}
  \item L($\epsilon$) =  \{``''\}
  \item union \hfill L(A + B) = \{a $|$ a $\in$ L (A)\} $\cup$ \{b $|$ b $\in$ L(B)\}
  \item Concat \hfill L(AB) = \{ab $|$ a $\in$ L(A) $\wedge$  b $\in$ L(B) \}
  \item Iteration a.k.a Kleene closure \hfill L(A*) = $\cup_{i \geq 0} L(A^i)$
  \end{enumerate}

  Why is L important?

  \begin{enumerate}
  \item More E than M: many ways to express same meaning
  \item Makes clear what is syntax and what is semantics : important (consider
    how Roman numerals and current number system have same meaning, but
    different notation)

    This allows us to separate the two and consider different ways of
    representation, allowing us to explore different notations
  \end{enumerate}

\end{enumerate}
}
\end{homeworkProblem}

\newpage
\begin{homeworkProblem}{\textbf{Lexical Analysis - Lexical Specifications}}

\begin{enumerate}
\item Strings
  \begin{enumerate}
  \item if... then... else...
  \item \textbf{Two equivalent forms} \hfil (concat then union)

    `i' 'f' + 'e' 'l' 's' 'e'

    'if' + 'else'
  \end{enumerate}
\item Digits
  \begin{enumerate}
  \item digit = '0' + '1' + ... + '9' \hfil Let digits refer to regexp
    containing all above

    - Refers to single digit
  \item digit$^*$

    - denotes almost all digits, except the first letter could be
    ``'',

    - to fix: digit digit$^*$ or digit+ (based on standard regexp)
  \end{enumerate}

\item Identifiers
  \begin{enumerate}
  \item Strings of letters or digits starting with an letter

    letter = 'a' + 'b' + ... + 'z' + 'A' + ... + 'Z' == [a-zA-Z]

  \item Identifier: letter(letter + digit)$^*$
  \end{enumerate}

\item Whitespace
  \begin{enumerate}
  \item non-empty sequence of blanks, newlines, and tabs
  \item blank : ' '
  \item newlines : '\textbackslash n'
  \item tabs : '\textbackslash t'
  \end{enumerate}
\end{enumerate}


\textbf{Constructing our lexer: Lexical Specification}


\begin{enumerate}
\item Write a rexp for the lexemes of each token class
  \begin{center}
    \begin{tabular}{|| c | c||}
      \hline
      Token Class & Lexeme\\
      \hline
      \hline
      Number & digit$^+$  \\
      \hline
      keyword & ``if'' + ``else'' \\
      \hline
      identifier & letter(letter + digit)$^*$\\
      \hline
      OpenPar & ``(''\\
      \hline
    \end{tabular}
  \end{center}

\item Construct R, matching all lexems for all tokens
  \begin{align}
    R &= Keyword + identifier + Number + ...\\
    &= R_1 + R_2 + ...
  \end{align}

  Which is the union of all the lexems from Step 1

\item Let input be x$_1$, ... x$_n$
  \begin{align}
    \text{For} i \leq i \leq n:& \\
                         & if (x_1,...,x_i \in L(R)): \text{remove} x_1,...x_i\\
  \end{align}
  \begin{enumerate}
  \item Maximal munch:
    \begin{enumerate}
    \item for some i $\neq$ j, and valid (x$_1$,...x$_i$) and valid (x$_1$, ..., x$_j$)
      which do we take?
    \item We take max(i, j)
    \end{enumerate}
  \item Which token?
    \begin{enumerate}
    \item Recall R = R$_1$ + ... + R$_n$
    \item given some valid string (x$_1$, ..., x$_n$) s.t $\in$ L(R$_i$) and $\in$
      L(R$_j$)
    \item typically just order s.t when pattern matching we choose the one listed
      first. \textbf{Priority ordering}
    \item E.g ``if'' is both in the language of keywords and identifiers, but we
      order it s.t keywords is first
    \end{enumerate}
  \item What if no rule matches?
    \begin{enumerate}
    \item x$_1$, ..., x$_i$ $\notin$ L(R) ??
    \item Define an Error string, s.t [all strings that are not matched]
    \item Has lowest priority in \textbf{priority ordering}
    \end{enumerate}
  \end{enumerate}
\end{enumerate}

\end{homeworkProblem}

\begin{homeworkProblem}{\textbf{Finite Automata}}

  \begin{enumerate}
  \item   Can be seen as analagous to langages, where depending on the string, we
    can say if it's in the language or not
  \item note: final state has two circles. Does not have self loop because if we
    get a string that repeats last character, it might not be in the language
    (which self loop would imply)
  \item \textbf{Termination}
    \begin{enumerate}
    \item String is finished, but not in accepting state
    \item Is in a ``dead end''
    \end{enumerate}
  \end{enumerate}
\end{homeworkProblem}

\begin{homeworkProblem}{\textbf{Regexps into NFAs}}

  \begin{enumerate}
  \item NFA = non-deterministic finite automaton
  \item For each kind of rexp, build a NFA
    \begin{enumerate}
    \item $\epsilon$: \hfill input $\rightarrow^{\epsilon}$ (fin)
    \item A: \hfill input $\rightarrow^{A}$ A(fin)
    \item AB: \hfill input $\rightarrow^{A}$ A(not fin) $\rightarrow^{\epsilon}$ B(fin)
    \item A + B: $\epsilon$ either to A or B then $\epsilon$ to fin
    \item AB and A+B as an NFA

      \includegraphics[width=13cm]{example_3}

    \item \textbf{Example: } (1 + 0)$^*$1

      \includegraphics[width=13cm]{example_1}
    \end{enumerate}
  \end{enumerate}
\end{homeworkProblem}

\begin{homeworkProblem}{\textbf{NFA to DFA}}

  \textbf{$\epsilon$-closure}
  \begin{enumerate}
  \item Defn: all states reachable by consuming an $\epsilon$ from current state.

    \textbf{Example}

    \includegraphics[width=13cm]{example_2}
  \end{enumerate}

  \textbf{NFA Details}
  \begin{enumerate}
  \item NFA may be in many states at once because non-deterministic
  \item How many states?

    N states

    $|S|$ $\leq$ N
  \item How many subsets?

    2$^N$ - 1 subsets \hfill May be many, but is \textit{finite}

  \item Comparison between NFA and DFA
    \begin{center}
      \begin{tabular}{|c | c| c|}
        \hline
        \hline
        Automata & NFA& DFA \\
        \hline
        \hline
        States & S & subsets of S (except empty set) \\
        Start & s $\in$ S & $\epsilon$-closure(s) \\
        Final & F $\subset$ S & \{X $|$ X $\cap$ F $\neq$ $\emptyset$\} \\
        Transition function & a(X) = \{y $|$ x $\in$ X $\wedge$ X $\rightarrow^a$ Y\} & X $\rightarrow^a$ Y if Y = $\epsilon$-closure(a(X)) \\
        \hline
      \end{tabular}
    \end{center}


  \item \textbf{NFA to DFA}
    \begin{enumerate}
    \item Begin with the start state, then determine the set of states in the
      $\epsilon$ closure
    \item Consider all possible transitions from the ``blob'', and generate new blobs
    \item Enumerate all the blobs
    \item Finish
    \end{enumerate}

    \includegraphics[width=13 cm]{example_4}
  \end{enumerate}

  The reason we defined the final state that way is that we don't care about any
  specific final state, we just want to get to a final state.

\end{homeworkProblem}


\begin{homeworkProblem}{\Large \textbf{Quiz Hints}}

  \begin{enumerate}
  \item First ignore the $\epsilon$ then find the number of possible strings. Then,
    consider with 1, 2, 3 and 4 epsilons.
  \item Consider each ``level'' to be another rule where it goes from level to
    level. Note, it does not loop on itself so if it's not fully eaten up by the
    end it's not accepted - this is in line with what happens if you try to
    construct a DFA
  \item Remember priority. Also, when they say tokenize here, they mean tokenize
    the entire thing (so the output of tokenization is the same as the input)
  \item Just trace it
  \item I considered the start state to be 1 state as well
  \item Remember + means 1 or more and * means 0 or more
  \item -
  \item -
  \item -
  \item I honestly don't remember how I did this without just working it out and
    spending a lot of time on this, sorry
  \item Read the definition
  \item Read the definition (the name itself is helpful here: non-deterministic)
  \end{enumerate}

\end{homeworkProblem}

\end{document}
