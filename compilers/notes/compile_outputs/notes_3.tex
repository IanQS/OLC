%%%%%%%%%%%%%%%%%%%%%%%%%%%%%%%%%%%%%%%%%
% Structured General Purpose Assignment
% LaTeX Template
%
% This template has been downloaded from:
% http://www.latextemplates.com
%
% Original author:
% Ted Pavlic (http://www.tedpavlic.com)
%
% Note:
% The \lipsum[#] commands throughout this template generate dummy text
% to fill the template out. These commands should all be removed when
% writing assignment content.
%
%%%%%%%%%%%%%%%%%%%%%%%%%%%%%%%%%%%%%%%%%

%----------------------------------------------------------------------------------------
%	PACKAGES AND OTHER DOCUMENT CONFIGURATIONS
%----------------------------------------------------------------------------------------

\documentclass{article}

\usepackage{fancyhdr} % Required for custom headers
\usepackage{lastpage} % Required to determine the last page for the footer
\usepackage{extramarks} % Required for headers and footers
\usepackage{graphicx} % Required to insert images
\usepackage{lipsum} % Used for inserting dummy 'Lorem ipsum' text into the
\usepackage{ textcomp }

\usepackage{hyperref}
\hypersetup{
    colorlinks=true,
    linkcolor=blue,
    filecolor=magenta,
    urlcolor=cyan,
}
% template

\usepackage{amsmath}

% Margins
\topmargin=-0.45in
\evensidemargin=0in
\oddsidemargin=0in
\textwidth=6.5in
\textheight=9.0in
\headsep=0.25in

\linespread{1.1} % Line spacing

% Set up the header and footer
\pagestyle{fancy}
\lhead{\hmwkAuthorName} % Top left header
\chead{\hmwkClass\ (\hmwkTitle)} % Top center header
\rhead{\firstxmark} % Top right header
\lfoot{\lastxmark} % Bottom left footer
\cfoot{} % Bottom center footer
\rfoot{Page\ \thepage\ of\ \pageref{LastPage}} % Bottom right footer
\renewcommand\headrulewidth{0.4pt} % Size of the header rule
\renewcommand\footrulewidth{0.4pt} % Size of the footer rule

\setlength\parindent{0pt} % Removes all indentation from paragraphs

%----------------------------------------------------------------------------------------
%	DOCUMENT STRUCTURE COMMANDS
%	Skip this unless you know what you're doing
%----------------------------------------------------------------------------------------

% Header and footer for when a page split occurs within a problem environment
\newcommand{\enterProblemHeader}[1]{
\nobreak\extramarks{#1}{#1 continued on next page\ldots}\nobreak
\nobreak\extramarks{#1 (continued)}{#1 continued on next page\ldots}\nobreak
}

% Header and footer for when a page split occurs between problem environments
\newcommand{\exitProblemHeader}[1]{
\nobreak\extramarks{#1 (continued)}{#1 continued on next page\ldots}\nobreak
\nobreak\extramarks{#1}{}\nobreak
}

\setcounter{secnumdepth}{0} % Removes default section numbers
\newcounter{homeworkProblemCounter} % Creates a counter to keep track of the number of problems

\newcommand{\homeworkProblemName}{}
\newenvironment{homeworkProblem}[1][Topic \arabic{homeworkProblemCounter}]{ % Makes a new environment called homeworkProblem which takes 1 argument (custom name) but the default is "Problem #"
\stepcounter{homeworkProblemCounter} % Increase counter for number of problems
\renewcommand{\homeworkProblemName}{#1} % Assign \homeworkProblemName the name of the problem
\section{\homeworkProblemName} % Make a section in the document with the custom problem count
\enterProblemHeader{\homeworkProblemName} % Header and footer within the environment
}{
\exitProblemHeader{\homeworkProblemName} % Header and footer after the environment
}

\newcommand{\problemAnswer}[1]{ % Defines the problem answer command with the content as the only argument
\noindent\framebox[\columnwidth][c]{\begin{minipage}{0.98\columnwidth}#1\end{minipage}} % Makes the box around the problem answer and puts the content inside
}

\newcommand{\homeworkSectionName}{}
\newenvironment{homeworkSection}[1]{ % New environment for sections within homework problems, takes 1 argument - the name of the section
\renewcommand{\homeworkSectionName}{#1} % Assign \homeworkSectionName to the name of the section from the environment argument
\subsection{\homeworkSectionName} % Make a subsection with the custom name of the subsection
\enterProblemHeader{\homeworkProblemName\ [\homeworkSectionName]} % Header and footer within the environment
}{
\enterProblemHeader{\homeworkProblemName} % Header and footer after the environment
}

%----------------------------------------------------------------------------------------
%	NAME AND CLASS SECTION
%----------------------------------------------------------------------------------------

\newcommand{\hmwkClass}{Compilers} % Course/class
\newcommand{\hmwkTitle}{Week 3 - Parsing and Top-Down parsing} % Assignment title
\newcommand{\hmwkClassTime}{-} % Class/lecture time
\newcommand{\hmwkAuthorName}{Ian Quah (itq)} % Your name

%----------------------------------------------------------------------------------------
%	TITLE PAGE
%----------------------------------------------------------------------------------------

\title{
\vspace{2in}
\textmd{\textbf{\hmwkClass:\ \hmwkTitle}}\\
\vspace{3in}
}

\author{\textbf{\hmwkAuthorName}}

%----------------------------------------------------------------------------------------

\begin{document}

\maketitle
\newpage
%----------------------------------------------------------------------------------------
%	PROBLEM 1
%----------------------------------------------------------------------------------------

% To have just one problem per page, simply put a \clearpage after each problem

\begin{homeworkProblem}{\Large \textbf{Parsing - Introduction}}

  \problemAnswer{

    \textbf{Regular Languages}
    \begin{enumerate}
    \item Weakest formal languages - most languages aren't regular

      \{($^i)^i$ $|$ i $\geq$ 0 \}, the set of all balanced parens, which can't be
      represented as a regular language

    \item What \textbf{CAN} a regular language express?
      \begin{enumerate}
      \item count mod k

        can't calculate to some arbitrarily high value, like in balanced parens problem
      \end{enumerate}

    \item
      \begin{tabular}{c| c| c}
        Phase & Input & Output\\
        \hline
        Lexer & String of Chars & String of Tokens\\
        Parser & String of tokens (output of lexer) & Parse Tree\\
      \end{tabular}
  \end{enumerate}
}
\end{homeworkProblem}

\begin{homeworkProblem}{\Large \textbf{Context Free Grammars}}

  \begin{enumerate}
  \item \textbf{The Problem}

    \begin{enumerate}
    \item Not all strings of tokens are programs

      Thus, need a language for describing valid strings of tokens and method to
      distinguish invalid and valid strings of tokens

      Note:
      RegExp $\subset$ CFG

      A regular language can be recognized by a finite automaton. A context-free language requires a stack

    \item Programming Languages have a recursive structure, and CFGs are a
      natural notation for describing these structures.
    \end{enumerate}

  \item \textbf{What IS a CFG?}

    \begin{enumerate}
    \item A set of terminals, T
    \item A set of non-terminals, N
    \item A start symbol, S (S $\in$ N)
    \item A set of productions: (A symbol, followed by an arrow then a list of symbols)

      X $\rightarrow$ Y$_1$, ... Y$_n$

      X $\in$ N and $Y_i \in N \cup T \cup \{\epsilon\}$
    \end{enumerate}

  \item \textbf{CFG example: Balanced Parens Problem}

    \textbf{Productions / Rules}

    \problemAnswer{
      S $\rightarrow$ (S) \hfill S, our start state then becomes another state with two
      parens around it

      S $\rightarrow$ $\epsilon$
    }

    \textbf{Our States}

    N = \{S\}

    T = \{(, )\}

    S = start symbol \hfill in general the first production will specify the
    start symbol for that CFG

  \item \textbf{Productions as rules:}

    \begin{enumerate}
    \item Begin with a String that only has start symbol, S
    \item Replace any non-terminal X in start by RHS of some production
    \item Repeat (2) until there are no non-terminals
    \end{enumerate}

    If we consider a single step as $\alpha$, then a program can be thought of as

    $\alpha_0$ $\rightarrow$ $\alpha_1$ $\rightarrow$ ... $\rightarrow$ $\alpha_n$ \hfill equiv $\alpha_0$ $\rightarrow^*$ $\alpha_n$


  \item \textbf{CFG formal definition}

    Let G be a CFG with start symbol S, then L(G) of G is:

    \[\{ a_1,...,a_n | \forall i , a_i \in T, S\rightarrow^* a_1,...,a_n \}\]


  \end{enumerate}
\end{homeworkProblem}

\begin{homeworkProblem}{\Large \textbf{Derivations Part 1}}
  \begin{enumerate}
  \item \textbf{Derivation:} a sequence of productions
  \item \textbf{Representations}

    - A derivation can be drawn as a tree instead of a linear path

    - A derivation can be drawn as a tree

    \begin{enumerate}
    \item Start symbol is the root
    \item For a production, add the children
    \end{enumerate}


    \textbf{Example: }

    Grammar:  E $\rightarrow$ E + E $|$ E * E $|$ (E) $|$ id

    \includegraphics[width=13cm]{derivation_tree}

    LHS: Derivation intermediates (Left-most derivation)

    RHS: Parse tree of input string

  \item \textbf{Parse Trees}
    \begin{enumerate}
    \item Terminals at the leaves
    \item Non-terminals at interiors
    \item in-order traversal of tree is original input
    \item Parse tree shows association of operations, input string does not (in
      tree we see multiplication is more tight than addition)
    \item Left-most and right-most derivation produces same parse tree, but the
      main difference is the intermediate step
    \end{enumerate}

  \end{enumerate}
\end{homeworkProblem}

\begin{homeworkProblem}{\Large \textbf{Ambiguity in CFGs}}

  \begin{enumerate}
  \item \textbf{Ambiguity}: if there is more than one parse tree for some string
    (more than 1 left-most or right-most derivation)

    This is bad, because it means that the decision is left to the compiler

  \item Determining if it's ambiguous

    Construct a string, then work backwards and see if there is more than 1 way
    to get the tree

  \item \textbf{Removing ambiguity}

    id * id + id

    E $\rightarrow$ E' + E | E'

    E' $\rightarrow$ id * E' | id | (E) * E' | (E)

    rewrite to enforce precedence of * over +, by separating multiplication and
    + s.t * is handled by E'
  \end{enumerate}
\end{homeworkProblem}

\begin{homeworkProblem}{\Large \textbf{Error Handling}}

  \textbf{Requirements of a Compiler}
  \begin{enumerate}
  \item Report errors accurately
  \item Not slow down compilation of valid code
  \end{enumerate}

  \textbf{Error Handling methods}
  \begin{enumerate}
  \item Panic Mode
    \begin{enumerate}
    \item Simplest, most popular
    \item When an error is detected:

      High-level Algorithm
      \begin{enumerate}
      \item Discated tokens until one with a clear role is found
      \item continue from there
      \end{enumerate}

    \item Details
      \begin{enumerate}
      \item Looks for syncrhonizing tokens
      \item typically statement or expression terminators
      \end{enumerate}
    \end{enumerate}

  \item Error productions
    \begin{enumerate}
    \item Specify known common mistakes in the grammar
    \item E.g :

      Write 5x instead of 5 * x

      Add the production E $\rightarrow$ ... $|$ E E \hfill which makes it now well-formed

    \end{enumerate}

  \item Automatic Error correction

    Idea: find some correct ``nearby'' program
    \begin{enumerate}
    \item Try token insertion and deletion
    \item Or do an exhaustive search
    \end{enumerate}

    Disadvantages
    \begin{enumerate}
    \item Hard to implement
    \item Storage of all the state is expensive
    \item Final output is not necessarily what the programmer intended. Safer to
      just error out
    \end{enumerate}
  \end{enumerate}
\end{homeworkProblem}

\begin{homeworkProblem}{\Large \textbf{Abstract Syntax Trees}}

  \textbf{Why not use the Parse Tree}
  \begin{enumerate}
  \item Contains a lot of redundant information.
  \end{enumerate}

  \begin{tabular}{c| c}
    \includegraphics[width=6cm]{ast} &  \includegraphics[width=6cm]{parse_tree}\\
  \end{tabular}

  See that in the parent of root nodes in parse tree, might as well just store
  the values themselves there, and that the parens were important in parsing but
  now they're not as important
\end{homeworkProblem}


\begin{homeworkProblem}{\Large \textbf{Recursive Descent}}

  \textbf{Top-Down Construction}
  \begin{enumerate}
  \item From the top
  \item From Left to Right
  \end{enumerate}

  Work our way down the productions, trying to match from the most general to
  most specific, then if it doesn't match at the root, we do backtracking up
  until the most general again

  \vspace{0.5 cm}

  {\Large \textbf{Recursive Descent Algorithm}}

  \begin{enumerate}
  \item A given token terminal

    bool term (TOKEN tok) \{return *next++==tok;\}

  \item The nth production of S:

    bool S$_n$() \{...\}

  \item Try all productions of S:

    bool S() \{...\}
  \end{enumerate}


  \problemAnswer{

    Example case:

    \begin{enumerate}
    \item     For production E $\rightarrow$ TOKEN

      bool E$_1$()\{return T();\}

    \item     For production E $\rightarrow$ T + E

      bool E$_2$\{return T() \&\& term(PLUS) \&\& E();\}

    \item For all productions of E(with backtracking)

      bool E()\{
      TOKEN *save = next;
      return (next =save, E$_1$()) $||$ (next=save, E$_2$();)
      \}


    \item Functions for non-terminal T

      bool T$_1$()\{return term(INT);\} \hfill T $\rightarrow$ int

      bool T$_2$() \{return term(INT) \&\& term(TIMES) \&\& T();\} \hfill
      T$\rightarrow$int * T

      bool T$_3$()\{return term(OPEN) \&\& E() \&\& term(CLOSE); \} \hfill T $\rightarrow$ (E)

      bool T()\{ TOKEN *save = next; return (T$_1$()) $||$ T$_2$() $||$ T$_3$); \}

    \end{enumerate}
}
    {\Large \textbf{Limitations of RDA}}

    \begin{enumerate}
    \item There is no backtracking for a production meaning even though we may have
      succeeded in matching a rule, we may not consume all the strings.
    \item Sufficient for grammars where for any non-terminal at most one
      production can succeed then.
    \item Thus, we need to introduce another method.
    \item * The method we were given earlier was given because it's mechanical
      and easy to do by hand.
    \end{enumerate}

\end{homeworkProblem}

\begin{homeworkProblem}{\Large \textbf{Left Recursion}}

  \textbf{Addresses the issues we brought up in the previous video}

  \begin{enumerate}
  \item  Consider the production S $\rightarrow$ S a
    \begin{enumerate}
    \item bool S$_1$() \{ return S() \&\& term(a);\}
    \item bool S() \{return S$_1$();\}
    \item S() goes into an infinite loop
    \end{enumerate}

  \item This is called a left-recursive grammar: it has a non-terminal S

    S $\rightarrow^+$ S $\alpha$ for some $\alpha$


  \item \textbf{Consider a more general production} S $\rightarrow$ S $\alpha$ $|$ $\beta$

    S generates all strings starting with a $\beta$ and followed by any number of
    $\alpha$

    S generates all strings starting with a $\beta$ followed by any number of $\alpha$s

  \item \textbf{The solution: rewriting using right-recursion}

    S $\rightarrow$ $\beta$ S'

    S' $\rightarrow$ $\alpha$ S' $|$ $\epsilon$

  \item \textbf{General form}

    \begin{enumerate}
    \item In General

      S $\rightarrow$ S $\alpha_1$ $|$ ... $|$ S $\alpha_n$ $|$ $\beta$ $|$... $\beta_m$

    \item All strings derived from S start with one of $\beta_1, ... \beta_m$ and
      continue with several instances of $\alpha_1, ... , \alpha_n$

    \item Rewrite as

      S $\rightarrow$ $\beta_1$ S' $|$ ... $|$ $\beta_m$ S'

      S' $\rightarrow$ $\alpha_1$S' $|$ ... $|$ $\alpha_n$ S' $|$ $\epsilon$
    \end{enumerate}
  \end{enumerate}
\end{homeworkProblem}


\begin{homeworkProblem}{\Large \textbf{Quiz hints}}

  \begin{enumerate}
  \item Expand the production rules, then think of binary
  \item Build off the previous one, then consider the number of times you can
    add epsilon as well as the number of unique strings it will generate
  \item Think of whether it would be possible to write a RegExp that would
    accept a string generated by the grammar
    \begin{enumerate}
    \item The answer was discussed in a video
    \item Construct a regexp
    \item My advice here is to start from the simplest case, then work your way
      backwards (I.e consider D, then C from D and so forth)

    \end{enumerate}
  \item unfortunately here there is no good trick that I know of. You need to
    substitute in all the values and see if it generates something
    left-recursive
  \item Substitutions
  \item Throw all you know about order of operations out and build different
    trees from there (the parenthesis rule still holds though). Write down the
    different configurations you generated - will be useful for next question
  \item Review the definition of
    \href{https://en.wikipedia.org/wiki/Operator_associativity}{associates
      left}, then consider each case given.
  \item -
  \item -
  \item Write out all the possible strings, then for each string identify the
    number of trees that can generate each of them
  \item I think of ambiguous as not being able to determine the order again, but
    if you keep the same ordering (hint hint), and the rules are uniform then
    you can generate the original (I \textit{think})
  \item Think of short circuiting, and review the RD algorithm thoroughly.
\end{enumerate}

\end{homeworkProblem}

\end{document}
