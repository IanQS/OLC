 %%%%%%%%%%%%%%%%%%%%%%%%%%%%%%%%%%%%%%%%%
% Structured General Purpose Assignment
% LaTeX Template
%
% This template has been downloaded from:
% http://www.latextemplates.com
%
% Original author:
% Ted Pavlic (http://www.tedpavlic.com)
%
% Note:
% The \lipsum[#] commands throughout this template generate dummy text
% to fill the template out. These commands should all be removed when
% writing assignment content.
%
%%%%%%%%%%%%%%%%%%%%%%%%%%%%%%%%%%%%%%%%%

%----------------------------------------------------------------------------------------
%	PACKAGES AND OTHER DOCUMENT CONFIGURATIONS
%----------------------------------------------------------------------------------------

\documentclass{article}

\usepackage{fancyhdr} % Required for custom headers
\usepackage{amsfonts}
\usepackage{bm}
\usepackage{lastpage} % Required to determine the last page for the footer
\usepackage{extramarks} % Required for headers and footers
\usepackage{graphicx} % Required to insert images
\usepackage{lipsum} % Used for inserting dummy 'Lorem ipsum' text into the
% template
\usepackage{amsmath}

% Margins
\topmargin=-0.45in
\evensidemargin=0in
\oddsidemargin=0in
\textwidth=6.5in
\textheight=9.0in
\headsep=0.25in

\linespread{1.1} % Line spacing

% Set up the header and footer
\pagestyle{fancy}
\lhead{\hmwkAuthorName} % Top left header
\chead{\hmwkClass\ (\hmwkTitle)} % Top center header
\rhead{\firstxmark} % Top right header
\lfoot{\lastxmark} % Bottom left footer
\cfoot{} % Bottom center footer
\rfoot{Page\ \thepage\ of\ \pageref{LastPage}} % Bottom right footer
\renewcommand\headrulewidth{0.4pt} % Size of the header rule
\renewcommand\footrulewidth{0.4pt} % Size of the footer rule

\setlength\parindent{0pt} % Removes all indentation from paragraphs

%----------------------------------------------------------------------------------------
%	DOCUMENT STRUCTURE COMMANDS
%	Skip this unless you know what you're doing
%----------------------------------------------------------------------------------------

% Header and footer for when a page split occurs within a problem environment
\newcommand{\enterProblemHeader}[1]{
\nobreak\extramarks{#1}{#1 continued on next page\ldots}\nobreak
\nobreak\extramarks{#1 (continued)}{#1 continued on next page\ldots}\nobreak
}

% Header and footer for when a page split occurs between problem environments
\newcommand{\exitProblemHeader}[1]{
\nobreak\extramarks{#1 (continued)}{#1 continued on next page\ldots}\nobreak
\nobreak\extramarks{#1}{}\nobreak
}

\setcounter{secnumdepth}{0} % Removes default section numbers
\newcounter{homeworkProblemCounter} % Creates a counter to keep track of the number of problems

\newcommand{\homeworkProblemName}{}
\newenvironment{homeworkProblem}[1][Topic \arabic{homeworkProblemCounter}]{ % Makes a new environment called homeworkProblem which takes 1 argument (custom name) but the default is "Problem #"
\stepcounter{homeworkProblemCounter} % Increase counter for number of problems
\renewcommand{\homeworkProblemName}{#1} % Assign \homeworkProblemName the name of the problem
\section{\homeworkProblemName} % Make a section in the document with the custom problem count
\enterProblemHeader{\homeworkProblemName} % Header and footer within the environment
}{
\exitProblemHeader{\homeworkProblemName} % Header and footer after the environment
}

\newcommand{\problemAnswer}[1]{ % Defines the problem answer command with the content as the only argument
\noindent\framebox[\columnwidth][c]{\begin{minipage}{0.98\columnwidth}#1\end{minipage}} % Makes the box around the problem answer and puts the content inside
}

\newcommand{\homeworkSectionName}{}
\newenvironment{homeworkSection}[1]{ % New environment for sections within homework problems, takes 1 argument - the name of the section
\renewcommand{\homeworkSectionName}{#1} % Assign \homeworkSectionName to the name of the section from the environment argument
\subsection{\homeworkSectionName} % Make a subsection with the custom name of the subsection
\enterProblemHeader{\homeworkProblemName\ [\homeworkSectionName]} % Header and footer within the environment
}{
\enterProblemHeader{\homeworkProblemName} % Header and footer after the environment
}

%----------------------------------------------------------------------------------------
%	NAME AND CLASS SECTION
%----------------------------------------------------------------------------------------

\newcommand{\hmwkClass}{Convex Optimization} % Course/class
\newcommand{\hmwkTitle}{CVX 101} % Assignment title
\newcommand{\hmwkClassTime}{-} % Class/lecture time
\newcommand{\hmwkAuthorName}{Ian Quah (itq)} % Your name

%----------------------------------------------------------------------------------------
%	TITLE PAGE
%----------------------------------------------------------------------------------------

\title{
\vspace{2in}
\textmd{\textbf{\hmwkClass:}\\
\textmd{\hmwkTitle}}\\
\vspace{3in}
}

\author{\textbf{\hmwkAuthorName}}

%----------------------------------------------------------------------------------------

\begin{document}

\maketitle

%----------------------------------------------------------------------------------------
%	PROBLEM 1
%----------------------------------------------------------------------------------------

% To have just one problem per page, simply put a \clearpage after each problem

\newpage
\begin{homeworkProblemName}{{\LARGE Week 2: Convex Sets}}

  \vspace{0.2 cm} \textbf{{\large Affine sets}}

  \begin{problemAnswer}{
      \begin{itemize}
      \item Defn: Contains the line through any two distinct points in
        a set

      \item Proof: Affine set contains line through any two points

        Given:

        1) $\{x | Ax = b\}$ (Set constructor)

        2) x = $\theta x_1$ + $(1 - \theta)x_2$ with $\theta \in \bm{R}$  (Definition of affine)

        3) A$x_1$ = b and A$x_2$ = b (Two points chosen arbitrarily)

        \begin{align*}
          Ax &= A(\theta x_1 + (1 - \theta)x_2) \tag{Fact 2}\\
          Ax &= \theta Ax_1 + A(1 - \theta)x_2\\
          Ax &= \theta b + (1 - \theta) b \tag{Fact 3}\\
             &= b
        \end{align*}

      \item Conversely, any affine set can be expressed as solution set of system of linear
        equations
      \end{itemize}
    }\end{problemAnswer}


  \vspace{0.2 cm} \textbf{{\large Convex sets}}

  \begin{problemAnswer}{

      \begin{itemize}
      \item  Line segment: between x$_1$ and x$_2$: all points

        \[x = \theta x_1 + (1 - \theta) x_2, \text{ with } 0 \leq \theta \leq 1\]

      \item Convex sets: contains line segment between any two points in the
        set

        \[x_1, x_2 \in \mathcal{C}, 0 \leq \theta \le 1 \implies \theta x_1 + (1 - \theta)x_2 \in \mathcal{C}\]
      \end{itemize}
    }\end{problemAnswer}

  \vspace{0.2 cm} \textbf{\large Convex combination and convex hulls}

  \begin{problemAnswer}{

      \begin{itemize}
        \item Convex combination of $x_1, ... , x_k$: any point x of the form
          \[x = \theta_1x_1 + .... + \theta_kx_k \text{ with }\]
            \[1) \theta_1 + ... + \theta_k = 1\]
            \[2) \theta_i \geq 0\]
        \item Convex hull conv S: set of all convex combinations of points in S

          E.g: Consider a square with dots on all 4 sides. The area within the
          set is the convex hull of the set S (which contains all 4 dots), as we
          can represent any point \textit{WITHIN} the area as some convex combination of the 4 points (or less, if a $\theta$ is 0)
      \end{itemize}
    }\end{problemAnswer}

  \clearpage

  \vspace{0.2 cm} \textbf{\large Convex cone}

  \begin{problemAnswer}{

      \begin{itemize}
        \item Conic (nonnegative) combination of $x_1, x_2$: any point x of the form
          \[x = \theta_1x_1 + \theta_2x_2 \text{ with }\]
          \[\theta_1, \theta_2 \geq 0\]
        \item Convex cone: set that contains all conic combinations of points in
          the set

        \item visually: a pie slice. Lies in a plane even in high dimensions,
          since we have 3 points

      \end{itemize}
    }\end{problemAnswer}

  \vspace{0.2 cm} \textbf{\large Hyperplanes and halfspaces}

  \begin{problemAnswer}{

      \begin{itemize}
        \item hyperplane: set of the form $\{x | a^Tx = b\}$ with a $\neq$ 0
        \item halfspace: set of the form $\{x | a^Tx \leq b\}$ with a $\neq$ 0
          such that a is the normal vector

        \item hyperplanes are affine and convex, whereas halfspaces are convex

        \item \textbf{Proof: Verifying why the halfspace is convex and NOT affine}

          1) $\{x | a^Tx \leq b\}$ (Set constructor)

          2) x = $\theta x_1$ + $(1 - \theta)x_2$ with $\theta \in \bm{R} $(Definition of affine)

          3) x = $\theta x_1$ + $(1 - \theta)x_2$ with $0 \leq \theta \leq 1$ (Definition of convex)

          4) $a^Tx_1$ $\leq$ b and $a^Tx_2$ $\leq$ b (Two points chosen arbitrarily)

          \begin{align*}
            \text{WTS: }a^Tx &\leq b\\
            a^T & = a^T(\theta x_1 + (1 - \theta)x_2) \tag{By definition in 2/3}\\
                             & = \theta a^T x_1 + (1 - \theta) a^T x_2\\
                             &\leq \theta b + (1 - \theta ) b \tag{Explanation 1} \\
                             &\leq b
          \end{align*}

          1) {\small We can put in the inequality, because both are non-negative. At
          this point, we cannot use 2 anymore, because if we chose any number
          larger than 1, we would end up with a negative now. Thus, it cannot be
        affine but is instead convex}

      \end{itemize}
    }\end{problemAnswer}

  \clearpage

  \vspace{0.2 cm} \textbf{\large Euclidean Ball and Ellipsoids}

  \begin{problemAnswer}{
      \begin{itemize}
      \item Euclidean ball: center x$_c$ and radius r:
        \begin{align*}
          B(x_c, r) &= \{x | \hspace{0.1 cm} ||x-x_c||_2 \leq r \tag{1}\\
                    &= \{x_c + ru | \hspace{0.1 cm} ||u||_2 \leq 1\} \tag{2}\\
        \end{align*}

        1) any point that is within r from the center is contained

        2) any point that is some multiple of the radius, where that multiple is
        between 0 and 1 is contained

      \item Ellipsoid: set of the form

        \begin{align*}
          E(x_c, r) &= \{x | (x - x_c)^TP^{-1}(x-x_c) \leq 1\} \tag{1}\\
                    &= \{x_c + Au | \hspace{0.1 cm} ||u||_2 \leq 1\} \tag{2}\\
        \end{align*}

        1) P $\in \bm{S}^n_{++}$ i.e P symmetric positive definite

        2) A is square and non-singular, and can be thought of as a
        transformation on the unit ball, u

        The ellipsoid is a generalization of the Euclidean ball. If P = $r^2$
        $\bm{I}$, we have that they are equal
      \end{itemize}
    }\end{problemAnswer}


  \vspace{0.2 cm} \textbf{\large Polyhedra or Polytope}

  \begin{problemAnswer}{
      \begin{itemize}
      \item Solution set of finitely many linear inequalities and equalities

        Ax $\prec$ b, Cx = d

        \[\text{Where } A \in \bm{R}^{m x n}, C \in \bm{R}^{p x n}, \text{ and } \prec
         \text{ is componentwise inequality}\]

      \item Is the intersection of finite number of halfspaces and hyperplanes
      \end{itemize}
    }\end{problemAnswer}


  \vspace{0.2 cm} \textbf{\large Positive Semidefinite Cone}

  \begin{problemAnswer}{
      \begin{itemize}
      \item $\bm{S}^n$ is set of symmetric n x n matrices
      \item $\bm{S}^n_+$ = \{X $\in \bm{S}^n | X \succeq 0$\}: positive semidefinite n x
        n matrices

        \[X \in \bm{S}^n_+ \iff z^TXz \geq 0 \forall z\]

        Note, that a linear combination of 

        $\bm{S}^n_+ \text{ is a convex cone} $

      \item $\bm{S}^n_{++}$ = \{X $\in \bm{S}^n | X \succ 0$\}: positive definite n x
        n matrices
      \end{itemize}
    }\end{problemAnswer}

  \clearpage

  {\Large Operations that preserve Convexity}
  

  \begin{problemAnswer}{
      \begin{itemize}
      \item Can always work:

        \[\text{arbitrary } x_1,  x_2 \in \mathcal{C}, 0 \leq \theta \leq  1 \implies \theta x_1 \text{ and } (1 - \theta)x_2 \text{ is in the set }\]
        
      \item Show that C is obtained from simple convex sets (hyperplanes,
        halfspaces, norm balls, ..) by operations that preserve convexity
        \begin{itemize}
        \item intersection
        \item affine functions
        \item perspective function
        \item linear-fractional functions
        \end{itemize}
      \end{itemize}
    }\end{problemAnswer}  

  \vspace{0.2 cm} \textbf{\large Preserving Convexity: Intersection}
  
  \begin{problemAnswer}{
      \begin{itemize}
      \item Intersection of any number of convex functions is convex

        \[\text{eg: } S = \{x \in \bm{R}^m | abs(p(t)) \leq 1 \text{ for } abs(t) \leq
          \frac{\pi}{3}\}\]

        where p(t) = x$_1$ cos t + x$_2$ cos 2t + ... + x$_m$ cos mt for m = 2

      \item Intuition behind example

         Repheasing S: \[S_t = \{x \in \bm{R}^m | |p(t)| \leq 1\}\]
         
         \begin{itemize}
         \item S$_t$ is a single point. It is the intersection of two parallel
           halfspaces (which are linear, as they are a linear combination of cos
           bla) with opposite orientations (
           hyperslab )

         \item As we modify t, we see that the slab produced by the two parallel lines
           rotates, and changes width.

         \item Thus, S is the intersection of those slabs.
           
         \end{itemize}  
      \end{itemize}
    }\end{problemAnswer}  

  \vspace{0.2 cm} \textbf{\large Preserving Convexity: Affine Function}
  
  \begin{problemAnswer}{
      \begin{itemize}
      \item Suppose f: $\bm{R}^n \to \bm{R}^m$ is affine (f(x) = Ax + b with A $\in
        \bm{R}^{m x n}, b \in \bm{R}^m$)

        \begin{itemize}
        \item The image of a convex set under f is convex

          \[S \subseteq \bm{R^n} \text{ convex } \implies f(S) = \{f(x) | x \in S\} \text{
              convex}\]

        \item The inverse image f$^{-1}$(C) of a convex set under f is convex

          \[C \subseteq \bm{R^m} \text{ convex } \implies f^{-1}(C) = \{x \in \bm{R^n} | f(x) \in C\} \text{ convex}\]
        \end{itemize}

      \item Examples
        \begin{itemize}
        \item scaling, translation, projection
        \item solution set of linear matrix inequality $\{x | x_1A_1 + ... +
          x_mA_m \succeq B\}$ with A$_i$, B $\in \bm{S}^P$
        \item hyperbolic cone $\{x | x^TPx \leq(c^Tx)^2, c^Tx \geq 0\}$ (with P $\in \bm{S}^n_+$)

      \end{itemize}
      \end{itemize}
    }\end{problemAnswer}  
  
\end{homeworkProblemName}


\end{document}
