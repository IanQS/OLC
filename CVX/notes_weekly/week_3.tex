 %%%%%%%%%%%%%%%%%%%%%%%%%%%%%%%%%%%%%%%%%
% Structured General Purpose Assignment
% LaTeX Template
%
% This template has been downloaded from:
% http://www.latextemplates.com
%
% Original author:
% Ted Pavlic (http://www.tedpavlic.com)
%
% Note:
% The \lipsum[#] commands throughout this template generate dummy text
% to fill the template out. These commands should all be removed when
% writing assignment content.
%
%%%%%%%%%%%%%%%%%%%%%%%%%%%%%%%%%%%%%%%%%

%----------------------------------------------------------------------------------------
%	PACKAGES AND OTHER DOCUMENT CONFIGURATIONS
%----------------------------------------------------------------------------------------

\documentclass{article}

\usepackage{pdfpages}
\usepackage{amssymb}
\usepackage{fancyhdr} % Required for custom headers
\usepackage{amsfonts}
\usepackage{bm}
\usepackage{lastpage} % Required to determine the last page for the footer
\usepackage{extramarks} % Required for headers and footers
\usepackage{graphicx} % Required to insert images
\usepackage{lipsum} % Used for inserting dummy 'Lorem ipsum' text into the
% template
\usepackage{amsmath}

% Margins
\topmargin=-0.45in
\evensidemargin=0in
\oddsidemargin=0in
\textwidth=6.5in
\textheight=9.0in
\headsep=0.25in

\linespread{1.1} % Line spacing

% Set up the header and footer
\pagestyle{fancy}
\lhead{\hmwkAuthorName} % Top left header
\chead{\hmwkClass\ (\hmwkTitle)} % Top center header
\rhead{\firstxmark} % Top right header
\lfoot{\lastxmark} % Bottom left footer
\cfoot{} % Bottom center footer
\rfoot{Page\ \thepage\ of\ \pageref{LastPage}} % Bottom right footer
\renewcommand\headrulewidth{0.4pt} % Size of the header rule
\renewcommand\footrulewidth{0.4pt} % Size of the footer rule

\setlength\parindent{0pt} % Removes all indentation from paragraphs

%----------------------------------------------------------------------------------------
%	DOCUMENT STRUCTURE COMMANDS
%	Skip this unless you know what you're doing
%----------------------------------------------------------------------------------------

% Header and footer for when a page split occurs within a problem environment
\newcommand{\enterProblemHeader}[1]{
\nobreak\extramarks{#1}{#1 continued on next page\ldots}\nobreak
\nobreak\extramarks{#1 (continued)}{#1 continued on next page\ldots}\nobreak
}

% Header and footer for when a page split occurs between problem environments
\newcommand{\exitProblemHeader}[1]{
\nobreak\extramarks{#1 (continued)}{#1 continued on next page\ldots}\nobreak
\nobreak\extramarks{#1}{}\nobreak
}

\setcounter{secnumdepth}{0} % Removes default section numbers
\newcounter{homeworkProblemCounter} % Creates a counter to keep track of the number of problems

\newcommand{\homeworkProblemName}{}
\newenvironment{homeworkProblem}[1][Topic \arabic{homeworkProblemCounter}]{ % Makes a new environment called homeworkProblem which takes 1 argument (custom name) but the default is "Problem #"
\stepcounter{homeworkProblemCounter} % Increase counter for number of problems
\renewcommand{\homeworkProblemName}{#1} % Assign \homeworkProblemName the name of the problem
\section{\homeworkProblemName} % Make a section in the document with the custom problem count
\enterProblemHeader{\homeworkProblemName} % Header and footer within the environment
}{
\exitProblemHeader{\homeworkProblemName} % Header and footer after the environment
}

\newcommand{\problemAnswer}[1]{ % Defines the problem answer command with the content as the only argument
\noindent\framebox[\columnwidth][c]{\begin{minipage}{0.98\columnwidth}#1\end{minipage}} % Makes the box around the problem answer and puts the content inside
}

\newcommand{\homeworkSectionName}{}
\newenvironment{homeworkSection}[1]{ % New environment for sections within homework problems, takes 1 argument - the name of the section
\renewcommand{\homeworkSectionName}{#1} % Assign \homeworkSectionName to the name of the section from the environment argument
\subsection{\homeworkSectionName} % Make a subsection with the custom name of the subsection
\enterProblemHeader{\homeworkProblemName\ [\homeworkSectionName]} % Header and footer within the environment
}{
\enterProblemHeader{\homeworkProblemName} % Header and footer after the environment
}

%----------------------------------------------------------------------------------------
%	NAME AND CLASS SECTION
%----------------------------------------------------------------------------------------

\newcommand{\hmwkClass}{Convex Optimization} % Course/class
\newcommand{\hmwkTitle}{CVX 101} % Assignment title
\newcommand{\hmwkClassTime}{-} % Class/lecture time
\newcommand{\hmwkAuthorName}{Ian Quah (itq)} % Your name

%----------------------------------------------------------------------------------------
%	TITLE PAGE
%----------------------------------------------------------------------------------------

\title{
\vspace{2in}
\textmd{\textbf{\hmwkClass:}\\
\textmd{\hmwkTitle}}\\
\vspace{3in}
}

\author{\textbf{\hmwkAuthorName}}

%----------------------------------------------------------------------------------------

\begin{document}

\maketitle

%----------------------------------------------------------------------------------------
%	PROBLEM 1
%----------------------------------------------------------------------------------------

% To have just one problem per page, simply put a \clearpage after each problem

\newpage
\begin{homeworkProblemName}{{\LARGE Week 3: Convex Functions}}

  \vspace{0.2 cm} \textbf{{\large Basic Properties and Examples}}

  \begin{problemAnswer}{
      \begin{itemize}
      \item Defn: $f: \bm{R}^n \rightarrow \bm{R}$ is convex if dom f is a convex set, and
        $$f(\theta x + (1 - \theta)y \leq \theta f(x) + (1 - \theta) f(y)) \text{ } \forall x, y \in \textbf{dom}
        f, 0 \leq \theta \leq 1 $$

        Aka, if you draw a bowl, with (x, f(x)) on the left side and (y, f(y))
        on the right side, the line connecting the two (the chord), is such that
        the chord is above the graph

      \item f is concave if -f is convex

      \item f is strictly convex if \textbf{dom} f is convex and

        $$ f(\theta x + (1 - \theta )y < \theta f(x) + (1 - \theta) f(y)) \text{ for }  x, y \in \textbf{dom}
        f, x \neq y,  0 < \theta < 1 $$


      \item Convex examples on $\bm{R}$
        \begin{itemize}
        \item affine: ax + b on $\bm{R} \text{ }\forall a, b \in \bm{R}$
        \item exponential: $e^{ax}$, for any a $\in \bm{R}$
        \item powers: $x^{\alpha}$ on $\bm{R}_{++}$, for $\alpha \geq 1 \text{ or } \alpha \leq 0$
        \item powers of absolute value: $|x|^{p}$ on $\bm{R}$, for $p \geq 1$
        \item Negative entropy: $x log x$ on $\bm{R}_{++}$
        \end{itemize}

      \item Concave examples on $\bm{R}$
        \begin{itemize}
        \item affine: ax + b on $\bm{R} \forall a, b \in \bm{R}$
        \item powers: $x^{\alpha}$ on $\bm{R}_{++}$, for $0 \leq \alpha \leq 1$
        \item logarithm: $log x$ on $\bm{R}_{++}$
        \end{itemize}

      \item Basically if you plot it you have a non-negative curvature (curves up)

      \item Examples on $\bm{R}^n \text{ and } \bm{R}^{m x n}$

        Affine functions are convex and concave; all norms are convex

        \textbf{Examples on R$^n$}

        \begin{itemize}
        \item affine function: f(x) = $a^Tx$ + b
        \item norms: $||x||_p$ = $(\sum_{i=1}^n |x_i|^p)^{\frac{1}{p}}$ for p $\geq$
          1; $||x||_{\infty} = max_k |x_k|$
        \end{itemize}

        \textbf{Examples on R$^{m x n}$}
        \begin{itemize}
        \item affine function:
          $$ f(X) = tr(A^TX) + b = \sum_{i=1}^m \sum_{j=1}^n A_{ij}X_{ij} + b $$
        \item spectral (maximum singular value) norms
          $$ f(X) = ||X||_2 = \sigma_{\text{max}}(X) = (\lambda_{\text{max}}(X^TX))^{\frac{1}{2}} $$
        \end{itemize}
      \end{itemize}
      }\end{problemAnswer}

    \newpage

    \begin{problemAnswer}{
        \begin{itemize}
        \item Convexity checking: restricting the convex function to a line
          \begin{itemize}
          \item f: $\bm{R}^n \rightarrow \bm{R} $ is convex iff the fn g: $\bm{R} \rightarrow
            \bm{R}$

            $$ g(t) = f(x + tv), \textbf{ dom } g = \{t | x + tv \in \textbf{ dom }
            f\} $$

            \begin{itemize}
            \item is convex (in t) for any x $\in \textbf{dom } f, v \in \bm{R^n}$

            \item can check convexity of f by checking convexity of functions of one
            variable

            \item The intuition here is that a function is convex iff when
              restricted to all lines it is convex
            \end{itemize}

          \item Example: f: $\bm{S}^n \rightarrow \bm{R} \text{ with } f(X) = log det X,
            \textbf{ dom } f = \bm{S}^n_{++}$

            \begin{align*}
              g(t) & = log det(X + tV) \tag{$X + tV = X^{\frac{1}{2}} (I + X^{-\frac{1}{2}} V X^{-\frac{1}{2}}) X^{\frac{1}{2}}$}\\
                   &= log det X + log det (I + tX^{-\frac{1}{2}}VX^{-\frac{1}{2}}) \tag{log det X constant, $\neg$ interesting}\\
              &= log det X + \sum_{i=1}^n log(1 + t \lambda_i) \tag{RHS is concave}\\
            \end{align*}

            where $\lambda_i$ are the eigenvalues of $X^{-\frac{1}{2}}VX^{-\frac{1}{2}}$

            g is concave in t (for any choice of X $\succ$0, V); hence f is concave
          \end{itemize}

        \item Extended-value extension

          extended-value extension $\tilde{f}$ of f is

          $$ \tilde{f}(x) = f(x), x \in \textbf{ dom } f, \tilde{f}(x) = \infty, x \notin
          \textbf{ dom } f, $$

          often simplifies notation; e.g

          $$ 0 \leq \theta \leq 1 \rightarrow \tilde{f}(\theta x + (1 - \theta)y) \leq \theta \tilde{f}(x) + (1 - \theta)\tilde{f}(y) $$

          (as an inequality in $\bm{R} \cup \{\infty\}$), means the same as the two conditions
            \begin{itemize}
            \item \textbf{dom } f is convex
            \item for x, y $\in \textbf{ dom }$ f,

              $$ 0 \leq \theta \leq 1 \implies f(\theta x + (1 - \theta)y) \leq \theta f(x) + (1 - \theta)f(y) $$
            \end{itemize}
        \end{itemize}
      }\end{problemAnswer}

    \newpage

    \begin{problemAnswer}{

      Conditions for convexity:

      \textbf{First-Order Condition}
      \begin{itemize}
      \item f is differentiable if \textbf{dom} f is open and the gradient

        $$ \nabla f(x) = ( \frac{\partial f(x)}{\partial x_1}, \frac{\partial f(x)}{\partial x_2}, ..., \frac{\partial
          f(x)}{\partial x_n}) $$, a column vector

        exists at each x $\in \textbf{dom} f$

      \item \textbf{1st-order condition: } differentiable f with convex domain
        is convex iff

        $$ f(y) \geq f(x) + \nabla f(x)^T(y-x) \forall x, y \in \textbf{ dom } f $$

        RHS: first order taylor expansion of f at the point x

        first-order approximation of f is a global underestimator
      \end{itemize}

      \textbf{Second-Order Condition}
      \begin{itemize}
      \item f is twice differentiable if \textbf{dom} f is open and the Hessian
        $\nabla^2$f(x) $\in \textbf{S}^n$ gradient

        $$ \nabla^2 f(x)_{ij} = \frac{\partial^2 f(x)}{\partial x_i \partial x_j}, i, j = 1,...,n$$

        exists at each x $\in \textbf{ dom }f$

      \item \textbf{2nd-order condition: } for twice differentiable f with convex domain
        is convex iff

        \begin{itemize}
        \item f is convex iff

          $$ \nabla^2 f(x) \succeq 0 \text{ for all x } \in \textbf{ dom } f, \text{the
            resulting matrix is PSD} $$

        \item if $ \nabla^2 f(x) \succeq 0$ for all x $\in \textbf{ dom } f $, then f is
          strictly convex
        \end{itemize}

      \end{itemize}

    }\end{problemAnswer}


  \begin{problemAnswer}{
    Examples

    \begin{itemize}
    \item quadratic functions: f(x) = $\frac{1}{2}x^TPx + q^Tx + r$ with P $\in
      \bm{S}^n$

      $$ \nabla f(x) = Px + q, \nabla^2 f(x) = P $$

      convex if P $\succeq$0

    \item Least-squares objective: f(x) = $|| Ax - b||_2^2$

      $$ \nabla f(x) = 2A^T(Ax - b), \nabla^2f(x) = 2A^TA $$

      convex (for any A)

    \item Quadratic-over-linear: f(x, y) = $\frac{x^2}{y}$

      $$ \nabla^2 f(x, y) = \frac{2}{y^3} \begin{bmatrix}y & -x\end{bmatrix}^T \begin{bmatrix}y & -x\end{bmatrix} \succeq 0$$

      convex for y $>$ 0

      if you plot this, you get the lorentz cone
    \end{itemize}


    }\end{problemAnswer}


\end{homeworkProblemName}


\end{document}
