%%%%%%%%%%%%%%%%%%%%%%%%%%%%%%%%%%%%%%%%%
% Structured General Purpose Assignment
% LaTeX Template
%
% This template has been downloaded from:
% http://www.latextemplates.com
%
% Original author:
% Ted Pavlic (http://www.tedpavlic.com)
%
% Note:
% The \lipsum[#] commands throughout this template generate dummy text
% to fill the template out. These commands should all be removed when
% writing assignment content.
%
%%%%%%%%%%%%%%%%%%%%%%%%%%%%%%%%%%%%%%%%%

%----------------------------------------------------------------------------------------
%	PACKAGES AND OTHER DOCUMENT CONFIGURATIONS
%----------------------------------------------------------------------------------------

\documentclass{article}

\usepackage{fancyhdr} % Required for custom headers
\usepackage{lastpage} % Required to determine the last page for the footer
\usepackage{extramarks} % Required for headers and footers
\usepackage{graphicx} % Required to insert images
\usepackage{lipsum} % Used for inserting dummy 'Lorem ipsum' text into the
% template
\usepackage{amsmath}

% Margins
\topmargin=-0.45in
\evensidemargin=0in
\oddsidemargin=0in
\textwidth=6.5in
\textheight=9.0in
\headsep=0.25in

\linespread{1.1} % Line spacing

% Set up the header and footer
\pagestyle{fancy}
\lhead{\hmwkAuthorName} % Top left header
\chead{\hmwkClass\ (\hmwkTitle)} % Top center header
\rhead{\firstxmark} % Top right header
\lfoot{\lastxmark} % Bottom left footer
\cfoot{} % Bottom center footer
\rfoot{Page\ \thepage\ of\ \pageref{LastPage}} % Bottom right footer
\renewcommand\headrulewidth{0.4pt} % Size of the header rule
\renewcommand\footrulewidth{0.4pt} % Size of the footer rule

\setlength\parindent{0pt} % Removes all indentation from paragraphs

%----------------------------------------------------------------------------------------
%	DOCUMENT STRUCTURE COMMANDS
%	Skip this unless you know what you're doing
%----------------------------------------------------------------------------------------

% Header and footer for when a page split occurs within a problem environment
\newcommand{\enterProblemHeader}[1]{
\nobreak\extramarks{#1}{#1 continued on next page\ldots}\nobreak
\nobreak\extramarks{#1 (continued)}{#1 continued on next page\ldots}\nobreak
}

% Header and footer for when a page split occurs between problem environments
\newcommand{\exitProblemHeader}[1]{
\nobreak\extramarks{#1 (continued)}{#1 continued on next page\ldots}\nobreak
\nobreak\extramarks{#1}{}\nobreak
}

\setcounter{secnumdepth}{0} % Removes default section numbers
\newcounter{homeworkProblemCounter} % Creates a counter to keep track of the number of problems

\newcommand{\homeworkProblemName}{}
\newenvironment{homeworkProblem}[1][Topic \arabic{homeworkProblemCounter}]{ % Makes a new environment called homeworkProblem which takes 1 argument (custom name) but the default is "Problem #"
\stepcounter{homeworkProblemCounter} % Increase counter for number of problems
\renewcommand{\homeworkProblemName}{#1} % Assign \homeworkProblemName the name of the problem
\section{\homeworkProblemName} % Make a section in the document with the custom problem count
\enterProblemHeader{\homeworkProblemName} % Header and footer within the environment
}{
\exitProblemHeader{\homeworkProblemName} % Header and footer after the environment
}

\newcommand{\problemAnswer}[1]{ % Defines the problem answer command with the content as the only argument
\noindent\framebox[\columnwidth][c]{\begin{minipage}{0.98\columnwidth}#1\end{minipage}} % Makes the box around the problem answer and puts the content inside
}

\newcommand{\homeworkSectionName}{}
\newenvironment{homeworkSection}[1]{ % New environment for sections within homework problems, takes 1 argument - the name of the section
\renewcommand{\homeworkSectionName}{#1} % Assign \homeworkSectionName to the name of the section from the environment argument
\subsection{\homeworkSectionName} % Make a subsection with the custom name of the subsection
\enterProblemHeader{\homeworkProblemName\ [\homeworkSectionName]} % Header and footer within the environment
}{
\enterProblemHeader{\homeworkProblemName} % Header and footer after the environment
}

%----------------------------------------------------------------------------------------
%	NAME AND CLASS SECTION
%----------------------------------------------------------------------------------------

\newcommand{\hmwkClass}{Mining Massive Datasets} % Course/class
\newcommand{\hmwkTitle}{Week 1: MapReduce} % Assignment title
\newcommand{\hmwkClassTime}{-} % Class/lecture time
\newcommand{\hmwkAuthorName}{Ian Quah (itq)} % Your name

%----------------------------------------------------------------------------------------
%	TITLE PAGE
%----------------------------------------------------------------------------------------

\title{
\vspace{2in}
\textmd{\textbf{\hmwkClass:\ \hmwkTitle}}\\
\vspace{3in}
}

\author{\textbf{\hmwkAuthorName}}

%----------------------------------------------------------------------------------------

\begin{document}

\maketitle

%----------------------------------------------------------------------------------------
%	PROBLEM 1
%----------------------------------------------------------------------------------------

% To have just one problem per page, simply put a \clearpage after each problem

\newpage

\begin{homeworkProblem}{}
Map, Reduce and program structure\vspace{10pt} % Question

\problemAnswer{ % Answer


  \begin{enumerate}
  \item   Map: (input elements) $\rightarrow$ (0 ++ key-value pairs)
  \item Reduce (key [V$_1$, V$_2$, .... , V$_n$]) $\rightarrow$ (0 ++ key-value pairs)

    \begin{enumerate}
    \item If the reducer function is associative and commutative (order doesn't
      matter) can push off computation to map function.
    \item This is the concept behind a combiner: (map $\rightarrow$ reduce) $\rightarrow$ reduce

      Examples:
      \begin{enumerate}
      \item       sum is associative and commutative
      \item average is not, however, if we store the aggregate as (sum, \# elems), we
        can reduce the number of tuples required to be transferred
      \end{enumerate}
    \end{enumerate}
  \item Functions provided by the user:
    \begin{enumerate}
    \item map
    \item reduce
    \item hash (optional): for user-specified grouping + specifying where to
      send jobs to
    \end{enumerate}

\item Structure of Program
    \begin{enumerate}
    \item Master node sends data to all compute nodes, then periodically
      pings them
    \item Dealing with Failure:
      \begin{enumerate}
      \item Only failure on master can bring the whole thing down
      \item Bc master can know when things are down (ping), it can detect if
        a node failed if it did, the entire calculation must be restarted (as
        data for reduce was stored in failed node)
      \item Master assigns the original task to whichever cluster frees up,
        then notifies reduce that the source
        of incoming data is changing
      \end{enumerate}
    \end{enumerate}
  \end{enumerate}
}
\end{homeworkProblem}

\begin{homeworkProblem}

  Relational algebra

  \problemAnswer{
    \begin{enumerate}
    \item Definitions:
      \begin{itemize}
      \item Attributes: Column headers
      \item Relation: table with attributes
      \item tuples: rows of the relation
      \item Schema: set of attributes of a relation
      \item R(A$_1$, A$_2$, ..., A$_n$):
        - Relation name: R

        - attributes: A$_1$,....A$_n$
      \end{itemize}
    \end{enumerate}

    \begin{enumerate}
    \item Relational Algebra Operations and definitions
      \begin{itemize}
      \item Selection:
        \begin{itemize}
        \item Apply condition C to each tuple.
        \item Returns only tuples that satisfy C
        \item Denoted $\sigma_C$(R)
        \end{itemize}
      \end{itemize}

      \begin{itemize}
      \item Projection:
        \begin{itemize}
        \item For some subset S of attributes ofR
        \item produce from each tuple only components for attributes in S
        \item Denoted $\pi_S$(R)
        \end{itemize}
      \end{itemize}

      \begin{itemize}
      \item Union, Intersection, Difference:
        \begin{itemize}
        \item Duh
        \end{itemize}
      \end{itemize}

      \begin{itemize}
      \item Natural Join:
        \begin{itemize}
        \item Denoted as R$_1$ $\bowtie$ R$_2$
        \item Given R$_1$, R$_2$, compare each pair of tuples
        \item If agree on ALL attributes common between the two:
          \begin{itemize}
          \item return tuple s.t
          \item \textbf{Cond 1}: has components for each attribute in either schema
          \item \textbf{Cond 2}: agrees with two tuples on all attribs
          \end{itemize}
        \item else:
          \begin{itemize}
          \item return None
          \end{itemize}

        \item Discussion below talks about Nat Joins, but also applies to
          \textit{equijoins}
          - equality covers equality of attributes that do not necessarily have
          the same name
        \end{itemize}
      \end{itemize}

      \begin{itemize}
      \item Grouping and Aggregation:
        \begin{itemize}
        \item Duh
        \end{itemize}
      \end{itemize}

    \end{enumerate}

  }

\end{homeworkProblem}


\end{document}
