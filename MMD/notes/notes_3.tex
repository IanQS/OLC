%%%%%%%%%%%%%%%%%%%%%%%%%%%%%%%%%%%%%%%%%
% Structured General Purpose Assignment
% LaTeX Template
%
% This template has been downloaded from:
% http://www.latextemplates.com
%
% Original author:
% Ted Pavlic (http://www.tedpavlic.com)
%
% Note:
% The \lipsum[#] commands throughout this template generate dummy text
% to fill the template out. These commands should all be removed when
% writing assignment content.
%
%%%%%%%%%%%%%%%%%%%%%%%%%%%%%%%%%%%%%%%%%

%----------------------------------------------------------------------------------------
%	PACKAGES AND OTHER DOCUMENT CONFIGURATIONS
%----------------------------------------------------------------------------------------

\documentclass{article}

\usepackage{fancyhdr} % Required for custom headers
\usepackage{lastpage} % Required to determine the last page for the footer
\usepackage{extramarks} % Required for headers and footers
\usepackage{graphicx} % Required to insert images
\usepackage{lipsum} % Used for inserting dummy 'Lorem ipsum' text into the
% template
\usepackage{amsmath}

% Margins
\topmargin=-0.45in
\evensidemargin=0in
\oddsidemargin=0in
\textwidth=6.5in
\textheight=9.0in
\headsep=0.25in

\linespread{1.1} % Line spacing

% Set up the header and footer
\pagestyle{fancy}
\lhead{\hmwkAuthorName} % Top left header
\chead{\hmwkClass\ (\hmwkTitle)} % Top center header
\rhead{\firstxmark} % Top right header
\lfoot{\lastxmark} % Bottom left footer
\cfoot{} % Bottom center footer
\rfoot{Page\ \thepage\ of\ \pageref{LastPage}} % Bottom right footer
\renewcommand\headrulewidth{0.4pt} % Size of the header rule
\renewcommand\footrulewidth{0.4pt} % Size of the footer rule

\setlength\parindent{0pt} % Removes all indentation from paragraphs

%----------------------------------------------------------------------------------------
%	DOCUMENT STRUCTURE COMMANDS
%	Skip this unless you know what you're doing
%----------------------------------------------------------------------------------------

% Header and footer for when a page split occurs within a problem environment
\newcommand{\enterProblemHeader}[1]{
\nobreak\extramarks{#1}{#1 continued on next page\ldots}\nobreak
\nobreak\extramarks{#1 (continued)}{#1 continued on next page\ldots}\nobreak
}

% Header and footer for when a page split occurs between problem environments
\newcommand{\exitProblemHeader}[1]{
\nobreak\extramarks{#1 (continued)}{#1 continued on next page\ldots}\nobreak
\nobreak\extramarks{#1}{}\nobreak
}

\setcounter{secnumdepth}{0} % Removes default section numbers
\newcounter{homeworkProblemCounter} % Creates a counter to keep track of the number of problems

\newcommand{\homeworkProblemName}{}
\newenvironment{homeworkProblem}[1][Topic \arabic{homeworkProblemCounter}]{ % Makes a new environment called homeworkProblem which takes 1 argument (custom name) but the default is "Problem #"
\stepcounter{homeworkProblemCounter} % Increase counter for number of problems
\renewcommand{\homeworkProblemName}{#1} % Assign \homeworkProblemName the name of the problem
\section{\homeworkProblemName} % Make a section in the document with the custom problem count
\enterProblemHeader{\homeworkProblemName} % Header and footer within the environment
}{
\exitProblemHeader{\homeworkProblemName} % Header and footer after the environment
}

\newcommand{\problemAnswer}[1]{ % Defines the problem answer command with the content as the only argument
\noindent\framebox[\columnwidth][c]{\begin{minipage}{0.98\columnwidth}#1\end{minipage}} % Makes the box around the problem answer and puts the content inside
}

\newcommand{\homeworkSectionName}{}
\newenvironment{homeworkSection}[1]{ % New environment for sections within homework problems, takes 1 argument - the name of the section
\renewcommand{\homeworkSectionName}{#1} % Assign \homeworkSectionName to the name of the section from the environment argument
\subsection{\homeworkSectionName} % Make a subsection with the custom name of the subsection
\enterProblemHeader{\homeworkProblemName\ [\homeworkSectionName]} % Header and footer within the environment
}{
\enterProblemHeader{\homeworkProblemName} % Header and footer after the environment
}

%----------------------------------------------------------------------------------------
%	NAME AND CLASS SECTION
%----------------------------------------------------------------------------------------

\newcommand{\hmwkClass}{Mining Massive Datasets} % Course/class
\newcommand{\hmwkTitle}{Week 3: Locality-Sensitive Hashing} % Assignment title
\newcommand{\hmwkClassTime}{-} % Class/lecture time
\newcommand{\hmwkAuthorName}{Ian Quah (itq)} % Your name

%----------------------------------------------------------------------------------------
%	TITLE PAGE
%----------------------------------------------------------------------------------------

\title{
\vspace{2in}
\textmd{\textbf{\hmwkClass:}\\
\textmd{\hmwkTitle}}\\
\vspace{3in}
}

\author{\textbf{\hmwkAuthorName}}

%----------------------------------------------------------------------------------------

\begin{document}

\maketitle

%----------------------------------------------------------------------------------------
%	PROBLEM 1
%----------------------------------------------------------------------------------------

% To have just one problem per page, simply put a \clearpage after each problem

\newpage
\begin{homeworkProblem}{\textbf{Finding Similar Sets}}
  \begin{enumerate}
  \item \textbf{Applications: }

  Given a body of docs, find pairs of documents with a lot of text in common
  \begin{enumerate}
  \item Mirror Sites, or approximate mirrors: don't want to show both in a
    search
  \item Plagiarism, including large quotations
  \item Similar news sites (cluster articles by ``same story
    '')
  \end{enumerate}


  \item \textbf{The process:} Shingling $\rightarrow$ Min-hashing $\rightarrow$ Locality-Sensitive hashing
    \begin{enumerate}
    \item \textbf{Shingling}: Convert docs, emails, etc. to sets
    \item \textbf{MinHasing}: convert large sets to short signatures, preserving
      similarity
    \item \textbf{Locality-Sensitive Hashing} focus on pairs of signatures likely
      to be similar
    \end{enumerate}


  \item \textbf{Shingles}

    \problemAnswer{
      A k-shingle (or k-gram) for a document is a sequence of k chars that appear
      in the document

      Example: k = 2, doc = abcab, set of 2-shingles = \{ab, bc, ca\}
    }

    \textbf{Shingles and Similarity}
    \begin{enumerate}
    \item Documents that are intuitively similar will have many shingles
    \item Changing a word only affects k-shingles within dist k from the word
    \item reordering paragraphs: affects 2k shingles that cross para boundaries
    \end{enumerate}

    \textbf{Shingles: Compression}
    \begin{enumerate}
    \item Compress long shingles (called tokens)
    \item Represent a doc by the tokens, the set of hash values of its k-shingles
    \end{enumerate}

\end{enumerate}
\end{homeworkProblem}

\begin{homeworkProblem}{\textbf{Min-Hashing}}

  \begin{enumerate}
  \item   \textbf{Jaccard Similarity}
    \begin{enumerate}
    \item   Size of intersection divided by size of the union
    \item Sim(C$_1$,C$_2$) = $\frac{| C_1 \cap C_2 |}{| C_1 \cup C_2 |}$
    \end{enumerate}

  \item \textbf{Sets to Boolean Matrices}
    \begin{enumerate}
    \item Rows = elems of universal set, e.g: set of all k-shingles
    \item Col = sets
    \item 1 in row e and col S iff e is a member of S
    \item Col similarity is Jaccard of sets of rows with at least 1 in either of
      the cols' rows
    \item Typical matrix is sparse
    \end{enumerate}

  \item \textbf{Consider the following example}

    \begin{tabular}{c|c}
      C$_1$ & C$_2$\\
      \hline
      0 & 1\\
      1 & 0\\
      1 & 1\\
      0 & 0\\
      1 & 1\\
      0 & 1
    \end{tabular}

    - 6 rows, but only 5 of them have at least a single 1 in them. Thus, denominator
    is 5

    - Sim(C$_1$,C$_2$ ) == 2/5 == 0.4

  \item \textbf{Minhashing}

    \begin{enumerate}
    \item     Defn: minhash function h(C) = \# of first row in which column C
      has 1
    \item     Use several independent hash functions to create signature for
      each column
    \item     Signature matrix: alternate repr for signatures - columns == sets and rows
    == minhash values, in order for that column

    \includegraphics[width=13cm]{minhashing}

  \item \textbf{Algorithm}
    \begin{enumerate}
    \item For each hash function
    \item permute the number of rows and then access the rows in that order
    \item access the cols (within the row) which have 1's in them and give them
      the index value (if they haven't been accessed before, else keep the min)
    \item Continue until we've gotten a hash for all columns
    \end{enumerate}

  \item  Similarity

    \begin{enumerate}
    \item Columns and sets: jacard similarity
    \item Signatures: fraction of components in which the two signatures agree
    \end{enumerate}

  \item \textbf{Simulating permutations w/o actually permuting}

    \begin{center}
    \includegraphics[height=5cm]{mh_1}

    \includegraphics[height=5cm]{mh_2}

    \includegraphics[height=5cm]{mh_3}
  \end{center}
  \end{enumerate}

\end{enumerate}

\end{homeworkProblem}

\end{document}
